\chapter*{Zusammenfassung}

\begin{otherlanguage}{ngerman}

Leistungssteigerungen in Computern werden seit dem ersten Jahrzehnt des 21. Jahrhunderts nicht mehr durch fortlaufende Erhöhung der Taktfrequenzen erreicht, sondern durch zunehmende Parallelisierung und Spezialisierung. Auf Grund dessen gewinnen heterogene Many-Core-Systeme zunehmend an Bedeutung. Somit müssen auch zunehmend Betriebssysteme Unterstützung bieten. Eine wichtige Rolle hierbei wird die Prozessablaufplanung und -zuordnung spielen, die für diese Systeme angepasst werden müssen. Aktuelle Architekturen für die Implementierung der Prozessablaufplanung können mit den resultierenden Herausforderungen nicht Schritt halten. Insbesondere ist es beim aktuellen Stand der Technik nicht möglich, auf Innovationen im Hardwarebereich rasch zu reagieren und die Prozessablaufplanung an die neuen Gegebenheiten anzupassen. Aus diesem Grund besteht der Beitrag zur Wissenschaft dieser Dissertation in einer neuartigen Architektur für die Prozessablaufplanung.

Diese Arbeit präsentiert einen neuen komponenten-basierten Ansatz zur Architektur der Prozessablaufplanung, welcher die Aufteilung des Planungsalgortithmuses in Komponenten erlaubt. Komponenten werden durch Pipes verbunden, welche eine Weiterentwicklung der bisher verwendeten Runqueues darstellen. Ferner werden Informationen unter den Komponenten durch ein publish-subscribe-basiertes Nachrichtensystem verbreitet. Komponenten können die Reihenfolge und die Verteilung der eingehenden Prozesse auf die ausgehenden Pipes bestimmen. Der Entwickler kann bestehende Komponenten wiederverwenden und neue Prozessablaufplanungsimplementierungen aus bestehenden erzeugen. Die vorgestellte Architektur erlaubt das einfache Auffinden von Flaschenhälsen durch ein expliziteres Layout des Prozessplaners verglichen zum monolithischen Ansatz. Ferner erlaubt die Architektur durch ihre expliziten Schnittstellen die Änderung der Planungsimplementierung zur Laufzeit. Der Prozessablaufplaner wird hierdurch optimal auf die Gegebenheiten des Systems angepasst.

Die Komponenten sind in ein Framework eingebettet, welches die transparente Entwicklung von Prozessplanungsimplementierungen ermöglicht. Der framework-basierte Ansatz erlaubt die einfache Einbindung bestehender Implementierungen ohne größere Änderungen in verschiedenen Laufzeitumgebungen. Diese Dissertation demonstriert die Machbarkeit des Ansatzes durch die Integration in zwei der größten offenen Betriebssystemkerne: Linux und FreeBSD. Basierend auf dieser Beispielimplementierung werden im Verlauf dieser Arbeit die Eigenschaften des Ansatzes evaluiert und diskutiert. Die Arbeit zeigt, dass der Ansatz für hunderte von Rechenkernen skaliert, sie quantifiziert den Overhead des Ansatzes und stellt die Vorteile wohl definierter Schnittstellen in diesem Anwendungsbereich am Beispiel vor. Schlussendlich werden die Vorzüge des Wechselns der Planungsstrategie zur Laufzeit dargestellt.

\end{otherlanguage}